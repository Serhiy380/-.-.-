\documentclass{article}
\usepackage[T2A]{fontenc}
\usepackage[utf8]{inputenc}
\usepackage[russian,ukrainian]{babel}
\usepackage{amsthm}

\begin{document}
\section{Початкові поняття}
\textbf{1}.За означенням $x < y$ означає, що $x \leq y$ та $x \not=y$. Довести, що у частково впорядкованій множині не існує елементу $x$ такого, що $x < x$ та що з $x < y$ та $y < z$ слідує $x < z$.

\begin{proof} Нехай існує такий елемент, що $x < x$. За означенням, має виконуватись $x \not= x$. Це є якраз таки протиріччям у найївній теорії множин, такого не може бути в ZFC. Протиріччя.
    
Друга частина.

З постановки задачі слідує, що $x \neq y$ та $y \neq z$ та $x \leq y$ та $y \leq z$. Хотілось би скористатися транзитивністю, але поки що з тих нерівностей не слідує, що $x \neq z$. Нехай $x = z$, тоді $x < y <x$. З цього слідує, що $x < x$. Вже було доведено, що це неможливо. Протиріччя. 
\end{proof}

\textbf{2}.Нехай бінарне відношення $<$ визначене як у вправі 1 та за визначенням $x \leq y$ означає, що $x < y$ або $x = y$. Показати, що відношення $\leq$ є частковим порядком.
\begin{proof}
    Оскільки $x \equiv x$, то з цього слідує рефлексивність.

З того, що $x < y$ та $y < x$ слідує антисиметричність за означенням цього відношення та протиріччям, що $x \neq x$ неможливе. Нехай $x < y$ та $y < z$. Нехай $y = z$. Тоді маємо $x < y = z$, тобто $x < z$. Якщо $x = y$, то знову ж слідує, що $x < z$.
\end{proof}

\textbf{3}. Довести наступну розширену властивість антисиметричності:
якщо $ x_0\leq x_1 \leq ... \leq x_{n-1} \leq x_0$, то $x_0 = x_1 = ... = x_{n-1}$.

\begin{proof}
    Індукція. $n = 1$. Рефлексивніть. Доведно.
    Нехай це доведено для $n-1$. Тоді маємо з транзитивності $ x_0 \leq x_1 \leq ... \leq x_{n-2} \leq x_0$, отже,   $x_0 = x_1 = ... = x_{n-2}$. $x_0 \leq x_{n-1} \leq x_0$. Отже, $x_0 \leq x_{n-1}$ та $x_{n-1} \leq x_0$. Отже, $x_0 = x_{n}$. Транзитивність рівності. Доведено. 
\end{proof}


\end{document}